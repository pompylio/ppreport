% Options for packages loaded elsewhere
\PassOptionsToPackage{unicode}{hyperref}
\PassOptionsToPackage{hyphens}{url}
%
\documentclass[
]{article}
\usepackage{amsmath,amssymb}
\usepackage{lmodern}
\usepackage{ifxetex,ifluatex}
\ifnum 0\ifxetex 1\fi\ifluatex 1\fi=0 % if pdftex
  \usepackage[T1]{fontenc}
  \usepackage[utf8]{inputenc}
  \usepackage{textcomp} % provide euro and other symbols
\else % if luatex or xetex
  \usepackage{unicode-math}
  \defaultfontfeatures{Scale=MatchLowercase}
  \defaultfontfeatures[\rmfamily]{Ligatures=TeX,Scale=1}
  \setmainfont[]{Arial}
\fi
% Use upquote if available, for straight quotes in verbatim environments
\IfFileExists{upquote.sty}{\usepackage{upquote}}{}
\IfFileExists{microtype.sty}{% use microtype if available
  \usepackage[]{microtype}
  \UseMicrotypeSet[protrusion]{basicmath} % disable protrusion for tt fonts
}{}
\makeatletter
\@ifundefined{KOMAClassName}{% if non-KOMA class
  \IfFileExists{parskip.sty}{%
    \usepackage{parskip}
  }{% else
    \setlength{\parindent}{0pt}
    \setlength{\parskip}{6pt plus 2pt minus 1pt}}
}{% if KOMA class
  \KOMAoptions{parskip=half}}
\makeatother
\usepackage{xcolor}
\IfFileExists{xurl.sty}{\usepackage{xurl}}{} % add URL line breaks if available
\IfFileExists{bookmark.sty}{\usepackage{bookmark}}{\usepackage{hyperref}}
\hypersetup{
  hidelinks,
  pdfcreator={LaTeX via pandoc}}
\urlstyle{same} % disable monospaced font for URLs
\usepackage[margin=0.5in]{geometry}
\usepackage{color}
\usepackage{fancyvrb}
\newcommand{\VerbBar}{|}
\newcommand{\VERB}{\Verb[commandchars=\\\{\}]}
\DefineVerbatimEnvironment{Highlighting}{Verbatim}{commandchars=\\\{\}}
% Add ',fontsize=\small' for more characters per line
\usepackage{framed}
\definecolor{shadecolor}{RGB}{248,248,248}
\newenvironment{Shaded}{\begin{snugshade}}{\end{snugshade}}
\newcommand{\AlertTok}[1]{\textcolor[rgb]{0.94,0.16,0.16}{#1}}
\newcommand{\AnnotationTok}[1]{\textcolor[rgb]{0.56,0.35,0.01}{\textbf{\textit{#1}}}}
\newcommand{\AttributeTok}[1]{\textcolor[rgb]{0.77,0.63,0.00}{#1}}
\newcommand{\BaseNTok}[1]{\textcolor[rgb]{0.00,0.00,0.81}{#1}}
\newcommand{\BuiltInTok}[1]{#1}
\newcommand{\CharTok}[1]{\textcolor[rgb]{0.31,0.60,0.02}{#1}}
\newcommand{\CommentTok}[1]{\textcolor[rgb]{0.56,0.35,0.01}{\textit{#1}}}
\newcommand{\CommentVarTok}[1]{\textcolor[rgb]{0.56,0.35,0.01}{\textbf{\textit{#1}}}}
\newcommand{\ConstantTok}[1]{\textcolor[rgb]{0.00,0.00,0.00}{#1}}
\newcommand{\ControlFlowTok}[1]{\textcolor[rgb]{0.13,0.29,0.53}{\textbf{#1}}}
\newcommand{\DataTypeTok}[1]{\textcolor[rgb]{0.13,0.29,0.53}{#1}}
\newcommand{\DecValTok}[1]{\textcolor[rgb]{0.00,0.00,0.81}{#1}}
\newcommand{\DocumentationTok}[1]{\textcolor[rgb]{0.56,0.35,0.01}{\textbf{\textit{#1}}}}
\newcommand{\ErrorTok}[1]{\textcolor[rgb]{0.64,0.00,0.00}{\textbf{#1}}}
\newcommand{\ExtensionTok}[1]{#1}
\newcommand{\FloatTok}[1]{\textcolor[rgb]{0.00,0.00,0.81}{#1}}
\newcommand{\FunctionTok}[1]{\textcolor[rgb]{0.00,0.00,0.00}{#1}}
\newcommand{\ImportTok}[1]{#1}
\newcommand{\InformationTok}[1]{\textcolor[rgb]{0.56,0.35,0.01}{\textbf{\textit{#1}}}}
\newcommand{\KeywordTok}[1]{\textcolor[rgb]{0.13,0.29,0.53}{\textbf{#1}}}
\newcommand{\NormalTok}[1]{#1}
\newcommand{\OperatorTok}[1]{\textcolor[rgb]{0.81,0.36,0.00}{\textbf{#1}}}
\newcommand{\OtherTok}[1]{\textcolor[rgb]{0.56,0.35,0.01}{#1}}
\newcommand{\PreprocessorTok}[1]{\textcolor[rgb]{0.56,0.35,0.01}{\textit{#1}}}
\newcommand{\RegionMarkerTok}[1]{#1}
\newcommand{\SpecialCharTok}[1]{\textcolor[rgb]{0.00,0.00,0.00}{#1}}
\newcommand{\SpecialStringTok}[1]{\textcolor[rgb]{0.31,0.60,0.02}{#1}}
\newcommand{\StringTok}[1]{\textcolor[rgb]{0.31,0.60,0.02}{#1}}
\newcommand{\VariableTok}[1]{\textcolor[rgb]{0.00,0.00,0.00}{#1}}
\newcommand{\VerbatimStringTok}[1]{\textcolor[rgb]{0.31,0.60,0.02}{#1}}
\newcommand{\WarningTok}[1]{\textcolor[rgb]{0.56,0.35,0.01}{\textbf{\textit{#1}}}}
\usepackage{graphicx}
\makeatletter
\def\maxwidth{\ifdim\Gin@nat@width>\linewidth\linewidth\else\Gin@nat@width\fi}
\def\maxheight{\ifdim\Gin@nat@height>\textheight\textheight\else\Gin@nat@height\fi}
\makeatother
% Scale images if necessary, so that they will not overflow the page
% margins by default, and it is still possible to overwrite the defaults
% using explicit options in \includegraphics[width, height, ...]{}
\setkeys{Gin}{width=\maxwidth,height=\maxheight,keepaspectratio}
% Set default figure placement to htbp
\makeatletter
\def\fps@figure{htbp}
\makeatother
\setlength{\emergencystretch}{3em} % prevent overfull lines
\providecommand{\tightlist}{%
  \setlength{\itemsep}{0pt}\setlength{\parskip}{0pt}}
\setcounter{secnumdepth}{-\maxdimen} % remove section numbering
\renewcommand{\contentsname}{}
\usepackage{pdflscape}
\newcommand{\blandscape}{\begin{landscape}}
\newcommand{\elandscape}{\end{landscape}}
\usepackage{tcolorbox}
\usepackage{caption}
\usepackage[utf8]{inputenc}
\captionsetup{labelformat=empty,textfont=bf}
\ifluatex
  \usepackage{selnolig}  % disable illegal ligatures
\fi

\author{}
\date{\vspace{-2.5em}}

\begin{document}

\begin{Shaded}
\begin{Highlighting}[]
\FunctionTok{pp\_title\_latex}\NormalTok{(}
  \AttributeTok{title =} \StringTok{"Receita Própria"}\NormalTok{,}
  \AttributeTok{subtitle =} \StringTok{"Execução Orçamentária e Financeira"}\NormalTok{,}
  \AttributeTok{bg\_color =} \StringTok{"000000"}\NormalTok{,}
  \AttributeTok{font\_color =} \StringTok{"ffffff"}\NormalTok{,}
  \AttributeTok{logo =} \StringTok{"\textasciitilde{}/projects/cgor{-}data/img/logo.png"}\NormalTok{)}
\end{Highlighting}
\end{Shaded}

\definecolor{bgcolor}{HTML}{000000} 
       \definecolor{ftcolor}{HTML}{ffffff}
       \begin{tcolorbox}[arc=0mm,colback=bgcolor,boxsep=3mm,colframe=bgcolor,colframe=white,sidebyside,righthand width=5cm]
        \color{ftcolor}{\vspace{0.2cm}{\LARGE{\textbf{Receita Própria}}}}
        
        \color{gray!20}{\rule{\linewidth}{0.2mm}}
        
        \color{ftcolor}{\textbf{Execução Orçamentária e Financeira}}
        
        \tcblower
        \includegraphics[height=6cm]{~/projects/cgor-data/img/logo.png}
       \end{tcolorbox}
       \color{black}

\begin{Shaded}
\begin{Highlighting}[]
\FunctionTok{pp\_content\_latex}\NormalTok{(}
  \AttributeTok{bg\_color =} \StringTok{"\#f2f2f2"}\NormalTok{,}
  \AttributeTok{border\_color =} \StringTok{"ffffff"}\NormalTok{, }
  \AttributeTok{border\_width =} \FloatTok{0.5}\NormalTok{)}
\end{Highlighting}
\end{Shaded}

\definecolor{colback}{HTML}{f2f2f2}
       \definecolor{colline}{HTML}{ffffff}
       \color{colline}{\rule{\linewidth}{0.5mm}}
       \begin{tcolorbox}[arc=0mm,colback=colback,colframe=white]
          
          \tableofcontents
      
          \end{tcolorbox}
       \color{colline}{\rule{\linewidth}{0.5mm}}
       \color{black}

\begin{Shaded}
\begin{Highlighting}[]
\FunctionTok{pp\_info\_latex}\NormalTok{(}
  \AttributeTok{x =} \FunctionTok{c}\NormalTok{(}\StringTok{"Exercício:"}\OtherTok{=}\StringTok{"2021"}\NormalTok{,}
        \StringTok{"Referência:"}\OtherTok{=}\FunctionTok{paste}\NormalTok{(}\StringTok{"2021{-}10{-}28"}\NormalTok{, }\StringTok{"(considerar dia útil anterior)"}\NormalTok{),}
        \StringTok{"Unidade Orçamentária"}\OtherTok{=}\StringTok{"26428 Instituto Federal de Brasília"}\NormalTok{,}
        \StringTok{"Responsável:"}\OtherTok{=}\StringTok{"Coordenação{-}Geral de Orçamento"}\NormalTok{,}
        \StringTok{"Contato:"}\OtherTok{=}\StringTok{"orcamento@ifb.edu.br"}\NormalTok{,}
        \StringTok{"Fonte:"}\OtherTok{=}\StringTok{"SIAFI"}\NormalTok{),}
  \AttributeTok{border\_color =} \StringTok{"f2f2f2"}\NormalTok{,}
  \AttributeTok{border\_width =} \FloatTok{0.2}\NormalTok{)}
\end{Highlighting}
\end{Shaded}

\definecolor{colline}{HTML}{f2f2f2}\definecolor{titlecolor}{HTML}{ffffff}\vfill\color{titlecolor}{\textbf{Parâmetros}}\linebreak\color{colline}{\rule{\linewidth}{0.2mm}}\linebreak\textcolor{black}{\text{\textbf{Exercício:} 2021}}\linebreak\textcolor{black}{\text{\textbf{Referência:} 2021-10-28 (considerar dia útil anterior)}}\linebreak\textcolor{black}{\text{\textbf{Unidade Orçamentária} 26428 Instituto Federal de Brasília}}\linebreak\textcolor{black}{\text{\textbf{Responsável:} Coordenação-Geral de Orçamento}}\linebreak\textcolor{black}{\text{\textbf{Contato:} orcamento@ifb.edu.br}}\linebreak\textcolor{black}{\text{\textbf{Fonte:} SIAFI}}\linebreak\color{colline}{\rule{\linewidth}{0.2mm}}\color{black}

\hypertarget{teste}{%
\section{Teste}\label{teste}}

teste

\end{document}
